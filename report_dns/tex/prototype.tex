\chapter{Prototype}
\section{Case}
A public school wishes to host its own DNS, with the following functionalities:
\begin{itemize}
\item Caching Name Server.
\item Forwarding to OpenDNS.
\item Blocking of certain sites.
\end{itemize}

\section{Analysis}

In your analysis you should at least address and/or include

\begin{itemize}
\item Overall diagram and description of the prototype
\item Relevance of the technology under consideration to your prototype
\item How the technology is included in your prototype
\item Definition of a small set of realistic use-cases and related
  functional requirements
\end{itemize}

\section{Implementation}
To host the DNS, BIND is used. A general guide to setup BIND is found in section \ref{sec:binddnsserver}.

\subsection{Caching}
Since the caching name server functionality is activated by default, this feature needs no further setup.

\subsection{Forwarding}
The only thing to change in the forwarding setup process is changing the use of Googles DNS to OpenDNS instead. In \emph{/etc/bind/named.conf.options} OpenDNS' address is inserted like this:

\texttt{//Forwarding to OpenDNS} \\
\texttt{forwarders \{208.67.222.222; 208.67.220.220\};}

\subsection{Blocking}

Furthermore blocking of sites is a requested feature. This can be done by zoning the domains, in the following way:

\section{Test}
