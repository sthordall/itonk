\documentclass[14pt]{article}

\begin{document}
\section{DNS Security Extension}
\label{DNS Security Extensions}
The original design of DNS did not include security, which make it vulnerable for hacking.
There are several ways to integrate security directly in the DNS, One of them is DNSSEC.
DNSEC  attempts to add security, while maintaining backwards compatibility.

\subsection{Threads}
There are several threads in DNS query/response transaction.
One of the simplest threads against DNS are various forms of packet interception: monkey-in-the-middle attacks, eavesdropping on requests combined with spoofed responses that beat the real response back to the resolver, and so forth.

Another interesting method of hacking is DNS spoofing or DNS cache poisoning attack.
The DNS spoofing is a computer hacking attack into DNS server's cache database, and change the name servers IP addresses. By changing the IP address, the user's traffic will be deverting to the hackers phishing site. 


\subsection{DNSSEC}
Domain Name System Security Extensions (DNSSEC) is a digital signature, which is designed to prevent hackers from changing the DNS process and direct users onto their own website to commit phishing. A long-term solution to this vulnerability is an end-to -end deployment of a security code. 

DNSSEC is a digital signature, which is set on the data to ensure that they are valid. It is not a technology that encrypts data, but it tells you about the validity of the address you are visiting. DNS entry is added a key and a signature that determines the validity of this key. It therefore does not alter the existing DNS protocol, but builds on it.

\subsection{Signed zones}
It is necessary to exploit it to the fullest, taking DNSSEC in use at each stage of the DNS process. 

DNSSEC incorporates a chain of digital signatures in the DNS hierarchy, with each level owns its own signature generating key. This means that the root zone signing a key .dk-zone and that .dk-zone signing key to iha.dk name server. ICANN, who controls the Root Server has DNSSEC in place, DK Hostmaster that controls all .dk domain names have DNSSEC in place, and many web hosts will also provide their users with DNSSEC, so the various links in the lookup process ensures each other.
\end{document}