
\section{DNS Security Extension}
\label{DNS Security Extensions}
The original design of DNS did not include security, which makes it vulnerable for hacking.
There are several ways to integrate security directly in the DNS, One of them is DNSSEC that attempts to add security, while maintaining backwards compatibility.

\subsection{Threads}
There are several threads in DNS a query/response transaction.
One of the simplest threads against DNS are various forms of packet interception: monkey-in-the-middle attacks, eavesdropping on requests combined with spoofed responses that beat the real response back to the resolver, and so forth.

Another interesting method of hacking is DNS spoofing or DNS cache poisoning attack.
The DNS spoofing is a computer hacking attack into DNS server's cache database and change the name servers IP-addresses. By changing the IP-address, the user's traffic will be diverting to the hackers phishing site.

\subsection{DNSSEC}
Domain Name System Security Extensions (DNSSEC) is a digital signature, which is designed to prevent hackers from changing the DNS process and direct users onto their own website to commit phishing. A long-term solution to this vulnerability is an end-to-end deployment of a security code. It is not a technology that encrypts data, but it tells you about the validity of the address you are visiting. DNS entry is added a key and a signature that determines the validity of this key. Therefore, it does not alter the existing DNS protocol, but builds on it.

\subsection{Signed zones}
It is necessary to exploit it to the fullest by taking DNSSEC in use at each stage of the DNS process.
Zones that implement DNSSEC are called signed zones, because they include digital signatures for resource records in their zone files,
served by DNSSEC-aware authoritative name servers.
DNSSEC incorporates a chain of digital signatures in the DNS hierarchy. On Each level in the hierarchy, the owner has their own signature generating key. 
E.g the iha.dk has the following layers:
Root-zone: ICANN who controls the Root server, they should have the DNSSEC in place.
.dk-zone: DK-Hostmaster who has the responsibility for all .dk, should have the DNSSEC in place.
web host: In this layer, the specific webhosting should have their own DNSSEC digital key generator.
With this method we will ensure the DNS from top to bottom.
If one or more of these zones doesn't have DNSSEC, we will be a vulnerable, and hackers can easy cheat us as mentioned previously.