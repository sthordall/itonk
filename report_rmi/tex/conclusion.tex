\chapter{Conclusion}
\section{Conclusion}

The project has shown, that RMI is a way of making a group of PC's and other java devices cooperate.
The RMI is a good and reliable way to connect a group of pc devices, and creates a good way to communicate. Through the leader election it can create a reliable(meaning: that it holds always has a leader, that is up to date, no matter what happens) leader selection, that secures that the network is always functional.

It has been proved in our prototype, that it is rather easy to setup and establish a RMI connection, and use it for communication. Since RMI is native to Java, configuration was minimal, and the code simple.

\section{Discussion}
The project work with the subject RMI has been more difficult compared to DNS and DDS.
Implementation of the algorithms was cumbersome, but the group succeeded.
The theoretical part of RMI was divided among the members of the group, whereas each group member would write in the report.

Making a prototype like this was relevant, because it increased the groups understanding, both for RMI as a technology and leader election as algorithms.

\section{Perspectives}
In RMI, remote interfaces are defined and registered in the RMI Registry with a corresponding ID to differentiate between identical interface implementations stored in the registry. This is very much how CORBA also works. Both the technologies share the idea of stubs and skeletons, which are handled by either ORB or the Registry.
