\chapter{Conclusion}
\section{Conclusion}
DDS consist of \emph{middleware} to create a portable bridge between systems.
There are lots of different ways of creating \emph{middleware}, while some are super secure, but less flexible, others are more flexible. \\
We have demonstrated the flexibility of the publish/subscribe paradigm. We show that it is a scalable system, and it is decoupled in time, space, and flow.

\section{Discussion}
As described before, \emph{middleware} works as a bridge and offers better scalability and flexibility to a system. Whether a system uses \emph{middleware} or not really depends on the size of the system. The bigger the system, the better it is to create a \emph{middleware} system. When taking into account which system, and what size of the system, it can easily become better to use \emph{middleware}, 
[OMFORMULERES]
because it offers easier understanding for developers than to interface to many different \emph{middleware} and software systems.\\

\section{Perspectives}
What defines a Data Distributed Service is the service that lies between applications. There are many approaches in the implementation of application software that applies to distribution of data. In general, a wide range of \emph{middlewares} exist, but when it comes to distribution of data, it can be defined loosely in two categories; human-time services in which it refers to human interaction and responses via web request servicing, and services that is performed within machine-time.

Standardized \emph{middlewares} exist in regard to web service such as SOAP (Simple Object Access Protocol), but in the latter category it is standardized differently, in which it relies on an association that relies on developers who contribute to a forum, so called Service Availability Forum (SAF).

When it comes to distribution of data in an embedded system, finding relevant ressources and solution for embedded systems are likely to be easy and easily accessible as \emph{middlewares} in the category of web services, which perhaps needs to be looked upon. 