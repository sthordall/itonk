\chapter{Conclusion}
\section{Conclusion}

\section{Discussion}

\section{Perspectives}
What defines a Data Distributed Service, is the service that lies between applications. There are many approaches in implementation of an application software that applies to distribution of data. There exist a wide range of middlewares in general, but when in it comes to distribution of data, it can be defined loosely in 2 categories; human-time services in which it refers to human interaction and responses via web request servicing, and services that is perfomed within machine-time.

There exist standardized middlewares in regard to webservice, such as SOAP (Simple Object Access Protocol), but in the latter category it is standardized differently, in which it relies on an association that relies on developers who contribute to a forum, so called Service Availability Forum (SAF).

When it comes to distribution of data in an embedded system, finding relevant ressource and solution for embedded systems is likely to be easy and easily accessible as middlewares in the category of webservices, which perhaps needs to be looked upon.