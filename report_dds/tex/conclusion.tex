\chapter{Conclusion}
\section{Conclusion}
DDS consist of Middleware, to create a portable bridge between systems. 
There are lots of different ways of creating MW, while some are super secure, but less flexible, others are wary flexible. \\
We have demonstrated the flexible publish/subscribe paradigm. We show that it is a scalable system, and it is decoupled in both time, space and flow.

\section{Discussion}
As escribed before, middleware works as a bridge, and offers better scalability, and flexibility to a system. Wether a system uses MW or not, really depends on the systems size. The bigger the system, the better it is to create a MW system. When take into account which system, and the size of the system, it can easily become better to use middleware, because it offers easier understanding for developers, than to interface to many different HW and SW systems.\\

\section{Perspectives}
What defines a Data Distributed Service, is the service that lies between applications. There are many approaches in implementation of an application software that applies to distribution of data. There exist a wide range of middlewares in general, but when in it comes to distribution of data, it can be defined loosely in 2 categories; human-time services in which it refers to human interaction and responses via web request servicing, and services that is perfomed within machine-time.

There exist standardized middlewares in regard to webservice, such as SOAP (Simple Object Access Protocol), but in the latter category it is standardized differently, in which it relies on an association that relies on developers who contribute to a forum, so called Service Availability Forum (SAF).

When it comes to distribution of data in an embedded system, finding relevant ressource and solution for embedded systems is likely to be easy and easily accessible as middlewares in the category of webservices, which perhaps needs to be looked upon.