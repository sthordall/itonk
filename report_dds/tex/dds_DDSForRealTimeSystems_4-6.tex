\section{Data distribution service for real-time systems}

\subsection{Quality of Service }

There are different ways to approach implementing a Distributed System. When a DDS is referred to as a data-centric middleware it means that the systems infrastructure takes more responsibility to reach its goal.

The fundamental goal is to get the right data, to the right place, at the right time.

In order to ensure a level of quality depending on the requirements, a Quality of Service exist to reach the goal.

Below picture contains an overview of different parameters that can be configured to meet the requirement depending on the need.

\begin{figure}[h]
\centering
\includegraphics[height=100mm, keepaspectratio]{img/dds_question_4-6/QoS_services}
\caption{QoS - Services}
\label{QoS - Services}
\end{figure}

\newpage

With these parameters a node or topic can be configured to meet certain requirements. 

An example where we have the following requirements:

\begin{itemize}
\item Single dataWriter, but multiple dataReaders
\item For every new dataReaders that joins the topic should get the last set of data
\item Every dataReaders should get all issued data
\end{itemize}

To ensure the requirement will be met, we can achieve it by making use of the services provided by the QoS. The requirement could be met by using the following services:

\begin{itemize}
\item History (Keep Last)
\begin{itemize}
\item To control how much data needs to be available
\end{itemize}
\item Reliability
\begin{itemize}
\item To ensure that data is delivered
\end{itemize}
\item Lifespan
\begin{itemize}
\item To know how and where data is stored
\end{itemize}
\end{itemize}

\subsection{Interface Definition Language}
An Interface Description Language defines the Type in a Topic where the Type is the definition of data.

It defines the objects interfaces via a standard specified as an Object Management Group (OMG) standard.

In general, IDL, describes an interface for a software componen independently of the language it is written in, in this way, achieving communication between software components regardless of the programming language.

\subseciton{Examples with Shape Demo and Java Examples}



\begin{figure}[h]
\centering
\includegraphics[height=100mm, keepaspectratio]{img/dds_question_4-6/QoS_services}
\caption{QoS - Services}
\label{QoS - Services}
\end{figure}

