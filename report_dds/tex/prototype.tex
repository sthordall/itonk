\chapter{Prototype}
\section{Analysis}
The prototype has been chosen to be a service that distributes news tagged with one or more predefined categories,
e.g. sports, economy, and science; the subscribers may only be interested in certain news
categories, and for our prototype, the topics will be IT and movies.

The prototype must at least involve:
1. Two publishers that publishes different kinds of data.
2. Two subscribers with different data needs.
3. Differentiated levels of quality of service.

The QoS has been chosen to be ...?B

\section{Implementation}
The implementation is split up into four main programs, two publishers and two subscribers, split into to topics. The implementation is topic based, and this is implemented as seen in figure figure\ref{TopicCreation}, this is done in both the subscriber and the publisher, and is how the connection between them or acknowledged.

begin{figure}[ht!]
\centering
\includegraphics[width=150mm]{img/TopicCreation.png}
\caption{Creation of topic.}
\label{TopicCreation}
\end{figure}

Next up is assigning the topic to either a datawriter, or a datareader. The publisher assigns a datawriter as seen in figure \ref{CreateDataWriter}, and the subscriber assigns a datareader as seen in figure \ref{CreateDataReader}.

begin{figure}[ht!]
\centering
\includegraphics[width=150mm]{img/CreateDataWriter.png}
\caption{Creation of Datawriter.}
\label{CreateDataWriter}
\end{figure}

begin{figure}[ht!]
\centering
\includegraphics[width=150mm]{img/CreateDataReader.png}
\caption{Creation of Datareader.}
\label{CreateDataReader}
\end{figure}

The publisher then publishes data on the datawriter, seen in figure \ref{DataWriterWrite}. This publishes the data on the given Topic.

begin{figure}[ht!]
\centering
\includegraphics[width=150mm]{img/DataWriterWrite.png}
\caption{Publishing data with the Datawriter.}
\label{DataWriterWrite}
\end{figure}

The subscriber extends the class \emph{DataReaderAdapter}, which enables it to implement the function \emph{on_data_available()}. This function is called when new data is published on the given topic. The specific implementation of this function is seen in figure \ref{OnDataAvailable}.

begin{figure}[ht!]
\centering
\includegraphics[width=150mm]{img/OnDataAvailable.png}
\caption{Implementation of \emph{on_data_available()}.}
\label{OnDataAvailable}
\end{figure}

Furthermore the QOS is setup in a configuration file, called \emph{USER_QOS_PROFILES.xml}. Examples of configuration of some of the policies are seen in the figure \ref{QOSProfiles}, in specifics it is the configuration for the datawriter. Policies as history, reliability, durability and resource limits are configured.

begin{figure}[ht!]
\centering
\includegraphics[width=150mm]{img/QOSProfiles.png}
\caption{QOS configuration of datawriter.}
\label{QOSProfiles}
\end{figure}

\section{Test}
